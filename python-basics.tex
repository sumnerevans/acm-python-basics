\documentclass{acm}

\usepackage{fontawesome}
\usepackage{etoolbox}
\usepackage{textcomp}
\usepackage[nodisplayskipstretch]{setspace}
\usepackage{xspace}
\usepackage{verbatim}
\usepackage{multicol}
\usepackage{soul}
\usepackage{attrib}

\usepackage{amsmath,amssymb,amsthm}

\usepackage[linesnumbered,commentsnumbered,ruled,vlined]{algorithm2e}
\newcommand\mycommfont[1]{\footnotesize\ttfamily\textcolor{blue}{#1}}
\SetCommentSty{mycommfont}
\SetKwComment{tcc}{ \# }{}
\SetKwComment{tcp}{ \# }{}

\usepackage{siunitx}

\usepackage{tikz}
\usepackage{pgfplots}
\usetikzlibrary{decorations.pathreplacing,calc,arrows.meta,shapes,graphs}

\AtBeginEnvironment{minted}{\singlespacing\fontsize{10}{10}\selectfont}
\usefonttheme{serif}

\makeatletter
\patchcmd{\beamer@sectionintoc}{\vskip1.5em}{\vskip0.5em}{}{}
\makeatother

% Math stuffs
\newcommand{\Z}{\mathbb{Z}}
\newcommand{\R}{\mathbb{R}}
\newcommand{\N}{\mathbb{N}}
\newcommand{\lcm}{\text{lcm}}
\newcommand{\Inn}{\text{Inn}}
\newcommand{\Aut}{\text{Aut}}
\newcommand{\Ker}{\text{Ker}\ }
\newcommand{\la}{\langle}
\newcommand{\ra}{\rangle}

\newcommand{\yournewcommand}[2]{Something #1, and #2}

\newenvironment{question}[1]{\par\textbf{Question #1.}\par}{}

\newcommand{\pmidg}[1]{\parbox{\widthof{#1}}{#1}}
\newcommand{\splitslide}[4]{
    \noindent
    \begin{minipage}{#1 \textwidth - #2 }
        #3
    \end{minipage}%
    \hspace{ \dimexpr #2 * 2 \relax }%
    \begin{minipage}{\textwidth - #1 \textwidth - #2 }
        #4
    \end{minipage}
}

\newcommand{\frameoutput}[1]{\frame{\colorbox{white}{#1}}}

\newcommand{\tikzmark}[1]{%
\tikz[baseline=-0.55ex,overlay,remember picture] \node[inner sep=0pt,] (#1)
{\vphantom{T}};
}

\newcommand{\braced}[3]{%
    \begin{tikzpicture}[overlay,remember picture]
        \draw [thick,decorate,decoration={brace,raise=1ex,amplitude=4pt},blue] (#2.south west-|T1.south west) -- node[anchor=west,left,xshift=-1.8ex,text=olive]{#3} (#1.north west-|T1.south west);
    \end{tikzpicture}
}

\newcommand{\make}{GNU \texttt{make}\xspace}

\title{Open Standards}
\author{Sumner Evans and Robby Zampino}
\institute{Mines Linux Users Group}

\begin{document}

\section{What is a Standard?}

\begin{frame}{What is a Standard?}

        A technical standard is an established norm or requirement for a
        repeatable technical task. It is usually a formal document that
        establishes uniform engineering or technical criteria, methods,
        processes, and practices.\footnote[frame]{
        \url{https://en.wikipedia.org/wiki/Technical_standard}}

        A de facto standard is a custom or convention that has achieved a
        dominant position by public acceptance or market forces (for
        example, by early entrance to the market).\footnote[frame]{
        \url{https://en.wikipedia.org/wiki/De_facto_standard}}

\end{frame}

\begin{frame}{What can a standard specify?}

        Technical standards can apply to all sort of things.
        \begin{itemize}
                        \pause
                \item Network protocols
                        \pause
                \item Filesystems
                        \pause
                \item File formats
                        \pause
                \item Peripherials
                        \pause
                \item APIs
                        \pause
                \item Connectors
        \end{itemize}


\end{frame}

\section{What is an \textit{Open} Standard?}

\begin{frame}{What is an \textit{Open} Standard?}

        An open standard is a standard is which a description of the standard
        is available publically. Open standards frequently also have open
        development processes, where anyone from the community can help shape
        the development of the standard. 

\end{frame}

\begin{frame}{Who defines Open Standards?}
        There are a ton of huge organizations that create open standards.
        The organizations handle definition, publicization, and  maintenance
        \pause
        \begin{itemize}
                \item IEEE\pause
                        \begin{itemize}
                                \item 802.11 Wi-Fi
                                \item 802.3 Ethernet
                        \end{itemize}\pause
                \item IETF\pause
                        \begin{itemize}
                                \item RFC793 TCP/IP
                                \item RFC768 UDP
                                \item RFC821 SMTP
                        \end{itemize}
        \end{itemize}
\end{frame}
\begin{frame}{Who defines \textit{Open} Standards?}
        There are also smaller organizations that focus on a single area.\pause
        \begin{itemize}
                \item USB-IF\pause
                        \begin{itemize}
                                \item Founded in 1983 by Compaq, Digital, IBM, Intel, Microsoft, NEC and Nortel
                                \item Focused on maintaining and advocating
                                        for USB related technologies
                        \end{itemize}\pause
                \item VESA\pause
                        \begin{itemize}
                                \item DisplayPort
                                \item Flat Display Mounting Interface (FDMI)
                        \end{itemize}
        \end{itemize}
\end{frame}

\begin{frame}{Why Do We Want Open Protocols?}
    Open protocols allow
    \begin{itemize}
        \item open development
        \item clean room implementation
        \item encourages competition
        \item auditability
    \end{itemize}
\end{frame}
% - Why it's better than closed protocols
%   - Can have multiple clients/servers all adhering to the protocol

\begin{frame}{Examples of Closed Standards}
        \begin{itemize}
                \item Skype
                        \pause
                \item Whatsapp (based on XMPP)
                        \pause
                \item Chromecast
                        \pause
                \item Slack
                        \pause
                \item Lightning Cables
                        \pause
                \item HDMI
                        \pause
                \item CUDA
        \end{itemize}
\end{frame}

\begin{frame}{A Few Terms}
    There are three main categories of architectures into which most
    applications that are not running on a single machine can be placed.
    \pause

    \begin{enumerate}
        \item \textbf{Client-server}: also known as \textit{centralized} or
            \textit{master-slave}, all producers and consumers connect to a
            single central server. (Example: GitLab)
        \item \textbf{Decentralized}: the producer and consumer connect directly
            to one another. (Example: BitTorrent)
        \item \textbf{Federated}: producers and consumers connect servers,
            and the servers connect to one another in a decentralized manner.
            (Example: Matrix)
    \end{enumerate}
    \pause

    Each can use open protocols, however federated and decentralized systems
    allow individuals or groups to own their own infrastructure and still
    connect to the broader userbase.
\end{frame}

\section{Cryptographic Standards}

\begin{frame}{Never Roll your own Cryptography}
    Tons of researchers and penetration testers have already verified the
    cryptographic security of all of the major cryptographic algorithms.
    \pause

    Open standards like the Advanced Encryption Standard (AES), RSA, OpenPGP,
    Elliptic Curve DSA, and all of the common hash functions underpin the
    security of the modern internet.
    \pause

    If these were not open standards, they would not be nearly as well
    widespread or trusted.
\end{frame}

\begin{frame}{The Signal Protocol}
    The Signal Protocol is a protocol developed by Open Whisper Systems for the
    Signal messenger.
    \pause

    Many closed-source programs including Facebook Messenger and WhatsApp
    \textit{claim} to have implemented this same protocol in their instant
    messenger applications.
    \pause

    The protocol is a combination of well-known cryptographic algorithms
    including triple Elliptic-Curve Diffie-Hellman and a \textit{Double Ratchet
        Algorithm}.
    \pause

    Due to the double ratchet, if an attacker cracks the key used to encrypt a
    single message, they will not be able to decipher subsequent or previous
    messages.
\end{frame}

\section{Application Layer Open Standards in Industry}

\begin{frame}{OpenRTB}
    The Interactive Advertising Bureau (IAB) publishes a specification for
    real-time bidding (RTB) on ad impression inventory.

    The protocol is a multi-layer protocol. This diagram shows the main
    components:\footnote[frame]{
        \url{https://github.com/InteractiveAdvertisingBureau/openrtb/blob/master/OpenRTB v3.0 FINAL.md}}

    \includegraphics[width=\textwidth]{./graphics/openrtb}
\end{frame}

\begin{frame}{Check 21}
    In 2003, the Federal Government passed the \textit{Check Clearing for the
        21st Century Act} (Check 21) which established the legality of using
    substitute checks and is the legal basis for remote deposit.
    \pause

    The Government gave no actual explanation of how it was supposed to work so
    the banking industry adopted a few of open standards for handling
    non-physical check deposits.
    \pause

    One such specification that was adopted is the \textit{Image Cash Letter}
    specification which dictates the file format for electronic check deposits.
\end{frame}

\begin{frame}{Image Cash Letter}
    Unfortunately, I'm very familiar with this file format due to working with
    it at one of my jobs.

    The Image Cash Letter format is a binary format with a collection of
    \textit{records}, each specifying something like file metadata, bundle
    metadata, check information, or check images.

    Each of the records has fixed width binary fields, and \textit{everything}
    (including numbers) is encoded as EBCDIC (except for the actual TIFF image
    data).
\end{frame}

\section{Application Layer Open Standards in Consumer Technology}

\begin{frame}{IRC}
    Internet Relay Chat (IRC) is an application layer protocol for textual
    communication. It is a client-server protocol where users can connect to a
    central server using a variety of clients (there are a \textit{ton} of
    clients)
    \pause

    The IRC protocol is available at https://tools.ietf.org/html/rfc1459
    The information here can be used to create a client for this protocol.
    It can also be used in a slightly stupider way.
\end{frame}


\begin{frame}{Matrix}
    Matrix is an open standard for ``secure, decentralized, real-time
    communication.''\footnote[frame]{https://matrix.org/}
    \pause

    It uses the same double ratchet algorithm as the Signal Protocol, but is
    designed with large rooms of thousands of people (much like IRC) in mind.
    \pause

    Additionally, it is designed to operate in a federated system; unlike Signal
    which uses a client-server model.
\end{frame}

\begin{frame}{Subsonic}
    Subsonic was an Open Source music server with an API for browsing songs,
    playlists, albums, and artists, among other features. Many clients were
    created around that API.
    \pause

    After the author closed sourced the application code, many people created
    forks of the last open source version of Subsonic.
    \pause

    Although the source code is still closed, Subsonic still publishes new
    versions of the Subsonic API specification, and the fork authors have
    implemented those endpoints in the forked versions of Subsonic.
\end{frame}

\begin{frame}{Case Study: Revel}
    Subsonic is a Java application, and all of the direct forks are also Java
    applications. This means they are slow and memory intensive.

    Robby has been working on an alternate server called \textit{Revel} which
    adheres to the Subsonic API spec.
    \pause

    Revel is written in Elixir and has no code in common with the original
    Subsonic server, yet you can \textit{basically} drop in Revel in place of Subsonic
    and clients will not notice a difference. Your server, however, will enjoy
    all the memory it get's back.
\end{frame}

\begin{frame}{Case Study: Sublime Music}
    All of the Subsonic clients for desktop suck. (There are a few good ones for
    mobile.) The web app is awful, and the existing desktop apps are hot
    garbage.

    Sumner has been working on an alternate client called \textit{Sublime Music}
    which can connect to any server which adheres to the Subsonic API.

    \begin{center}
        \includegraphics[width=0.57\textwidth]{./graphics/sublime-music}
    \end{center}
\end{frame}

% The subtopics would be how open protocols drive Industry, why they are
% superior to closed protocols, and a case study in how they can be good for
% consumers as well. Obviously that last one would be subsonic. I'm not sure
% about the order, but that's the general idea.

\end{document}
% Local Variables:
% TeX-command-extra-options: "-shell-escape"
% End:
