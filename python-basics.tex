\documentclass{acm}

\usepackage{fontawesome}
\usepackage{etoolbox}
\usepackage{textcomp}
\usepackage[nodisplayskipstretch]{setspace}
\usepackage{xspace}
\usepackage{verbatim}
\usepackage{multicol}
\usepackage{soul}
\usepackage{attrib}

\usepackage{amsmath,amssymb,amsthm}

\usepackage[linesnumbered,commentsnumbered,ruled,vlined]{algorithm2e}
\newcommand\mycommfont[1]{\footnotesize\ttfamily\textcolor{blue}{#1}}
\SetCommentSty{mycommfont}
\SetKwComment{tcc}{ \# }{}
\SetKwComment{tcp}{ \# }{}

\usepackage{siunitx}

\usepackage{tikz}
\usepackage{pgfplots}
\usetikzlibrary{decorations.pathreplacing,calc,arrows.meta,shapes,graphs}

\AtBeginEnvironment{minted}{\singlespacing\fontsize{10}{10}\selectfont}
\usefonttheme{serif}

\makeatletter
\patchcmd{\beamer@sectionintoc}{\vskip1.5em}{\vskip0.5em}{}{}
\makeatother

% Math stuffs
\newcommand{\Z}{\mathbb{Z}}
\newcommand{\R}{\mathbb{R}}
\newcommand{\N}{\mathbb{N}}
\newcommand{\lcm}{\text{lcm}}
\newcommand{\Inn}{\text{Inn}}
\newcommand{\Aut}{\text{Aut}}
\newcommand{\Ker}{\text{Ker}\ }
\newcommand{\la}{\langle}
\newcommand{\ra}{\rangle}

\newcommand{\yournewcommand}[2]{Something #1, and #2}

\newenvironment{question}[1]{\par\textbf{Question #1.}\par}{}

\newcommand{\pmidg}[1]{\parbox{\widthof{#1}}{#1}}
\newcommand{\splitslide}[4]{
    \noindent
    \begin{minipage}{#1 \textwidth - #2 }
        #3
    \end{minipage}%
    \hspace{ \dimexpr #2 * 2 \relax }%
    \begin{minipage}{\textwidth - #1 \textwidth - #2 }
        #4
    \end{minipage}
}

\newcommand{\frameoutput}[1]{\frame{\colorbox{white}{#1}}}

\newcommand{\tikzmark}[1]{%
\tikz[baseline=-0.55ex,overlay,remember picture] \node[inner sep=0pt,] (#1)
{\vphantom{T}};
}

\newcommand{\braced}[3]{%
    \begin{tikzpicture}[overlay,remember picture]
        \draw [thick,decorate,decoration={brace,raise=1ex,amplitude=4pt},blue] (#2.south west-|T1.south west) -- node[anchor=west,left,xshift=-1.8ex,text=olive]{#3} (#1.north west-|T1.south west);
    \end{tikzpicture}
}

\title{Python Basics}
\author{Sumner Evans}
\institute{Mines ACM}

\begin{document}

\begin{frame}{A Small Survey}
    Welcome everyone! I'd like to get to know everyone a bit more and get a feel
    for everyone's prior experience with programming and Python.

    \begin{itemize}[<+->]
        \item What year are you in?
        \item How many of you have programmed in any language before?
        \item How many of you have programmed in \textbf{Python} before?
    \end{itemize}
\end{frame}

\begin{frame}{Overview}
    \setbeamertemplate{section in toc}[sections numbered]
    \tableofcontents[hideallsubsections]
\end{frame}

\section{What is Python?}

\begin{frame}{A Bit of History}
    \begin{itemize}[<+->]
        \item Python first appeared in early 1991. \emph{This means that Python
            is older than Java and Ruby.}
        \item Guido van Rossum (GvR, the creator of Python) designed his
            language with \textbf{emphasis on readability}.
        \item Python was named after \emph{Monty Python's Flying Circus}.
        \item The language quickly gained popularity because of its appeal to
            long-time UNIX/C hackers\footnote{See the Jargon File on hackers
            vs.  crackers}.
    \end{itemize}
\end{frame}

\begin{frame}{Why Learn Python?}
    \begin{itemize}[<+->]
        \item Python is a \textit{general purpose, multi-paradigm} language
            meaning that it is very flexible and can be used in many different
            scenarios.
        \item Some of the main applications of Python in industry are web
            programming, data science, machine learning, automation scripting.
        \item Python is an easy language to learn.
        \item Python runs anywhere, and generally requires little setup compared
            to other languages.
    \end{itemize}
\end{frame}

\begin{frame}{A Note on Python 2 and Python 3}
    There are two main versions of Python: Python 2 and Python 3. As of earlier
    this year, Python 2 is no longer supported, so nobody should use it.
    Unfortunately, many projects and operating systems have not gotten with the
    times and are still reliant on Python 2.

    \pause
    Python 3 has many major advantages over Python 2 as it fixes many annoying
    inconsistencies with the older version.

    \pause
    For the purposes of this presentation, we will be talking \emph{strictly of
    Python 3}.
\end{frame}

\section{Programming Basics in Python}

\begin{frame}{Follow Along}
    You can either install Python on your machine or use an online Python
    environment such as \url{https://repl.it/languages/Python3}.

    Most of the things we will cover today can be done directly in the REPL
    (read-evaluate-print-loop) on the right, however you may want to write code
    in the file on the left and run it.
\end{frame}

\begin{frame}[fragile]{Storing Data}
    At its core, programming is about storing and manipulating \textit{data}.

    \pause
    In almost every programming language, there is a concept of a
    \textbf{variable} which \textit{stores} data.

    \pause
    In Python, you can create a variable using the following syntax:
    \begin{minted}{python}
        name = "Sumner"
        age = 22
        likes_acm = True
    \end{minted}

    \textit{If you have ever programmed in a language such as Java, you may
    notice that there is no special keyword for declaring a variable.}
\end{frame}

\begin{frame}[fragile]{Showing the Data}
    Storing data isn't any good unless you can actually use it for something
    useful. \pause One of the most basic things we can do with the data stored
    is print it out to the console.

    To print anything in Python, use the \texttt{print} function:
    \begin{minted}{python}
        name = "Sumner"
        age = 22
        print(name)
        print(age)
    \end{minted}

    \pause
    If you want to print multiple things at once, you can separate them with a
    comma:
    \begin{minted}{python}
        print(name, age)
    \end{minted}
\end{frame}

\begin{frame}{What Sorts of Data Can We Store?}
    There are many different \textit{types} of data that we can store in Python.
    Here are the most basic data types (primitives):

    \begin{itemize}
        \item \texttt{bool} --- either \texttt{True} or \texttt{False}
        \item \texttt{int} --- an integer
        \item \texttt{float} --- a real number\footnote[frame]{Not all real
            numbers can be represented as a floating point number, but that's
            not normally important.}
        \item \texttt{string} --- a sequence of characters\footnote[frame]{Note
            that unlike other languages, there is no \texttt{char} datatype.
            Chars are just one-character strings.}
    \end{itemize}
\end{frame}

\begin{frame}{Wh}
    (<>)
\end{frame}

\begin{frame}{ohea}
    * Variables
        * Data types
    * Basic operations on simple data types
    * If statements
    * For loops
    * Functions
        * Parameters
    * Union data types
\end{frame}

\end{document}
% Local Variables:
% TeX-command-extra-options: "-shell-escape"
% End:
