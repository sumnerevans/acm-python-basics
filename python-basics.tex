\documentclass{acm}

\usepackage{fontawesome}
\usepackage{etoolbox}
\usepackage{textcomp}
\usepackage[nodisplayskipstretch]{setspace}
\usepackage{xspace}
\usepackage{verbatim}
\usepackage{multicol}
\usepackage{soul}
\usepackage{attrib}

\usepackage{amsmath,amssymb,amsthm}

\usepackage[linesnumbered,commentsnumbered,ruled,vlined]{algorithm2e}
\newcommand\mycommfont[1]{\footnotesize\ttfamily\textcolor{blue}{#1}}
\SetCommentSty{mycommfont}
\SetKwComment{tcc}{ \# }{}
\SetKwComment{tcp}{ \# }{}

\usepackage{siunitx}

\usepackage{tikz}
\usepackage{pgfplots}
\usetikzlibrary{decorations.pathreplacing,calc,arrows.meta,shapes,graphs}

\AtBeginEnvironment{minted}{\singlespacing\fontsize{10}{10}\selectfont}
\setmainfont{Open Sans Light}
\usefonttheme{serif}

\makeatletter
\patchcmd{\beamer@sectionintoc}{\vskip1.5em}{\vskip0.5em}{}{}
\makeatother

% Math stuffs
\newcommand{\Z}{\mathbb{Z}}
\newcommand{\R}{\mathbb{R}}
\newcommand{\N}{\mathbb{N}}
\newcommand{\lcm}{\text{lcm}}
\newcommand{\Inn}{\text{Inn}}
\newcommand{\Aut}{\text{Aut}}
\newcommand{\Ker}{\text{Ker}\ }
\newcommand{\la}{\langle}
\newcommand{\ra}{\rangle}

\newcommand{\yournewcommand}[2]{Something #1, and #2}

\newenvironment{question}[1]{\par\textbf{Question #1.}\par}{}

\newcommand{\pmidg}[1]{\parbox{\widthof{#1}}{#1}}
\newcommand{\splitslide}[4]{
    \noindent
    \begin{minipage}{#1 \textwidth - #2 }
        #3
    \end{minipage}%
    \hspace{ \dimexpr #2 * 2 \relax }%
    \begin{minipage}{\textwidth - #1 \textwidth - #2 }
        #4
    \end{minipage}
}

\newcommand{\frameoutput}[1]{\frame{\colorbox{white}{#1}}}

\newcommand{\tikzmark}[1]{%
\tikz[baseline=-0.55ex,overlay,remember picture] \node[inner sep=0pt,] (#1)
{\vphantom{T}};
}

\newcommand{\braced}[3]{%
    \begin{tikzpicture}[overlay,remember picture]
        \draw [thick,decorate,decoration={brace,raise=1ex,amplitude=4pt},blue] (#2.south west-|T1.south west) -- node[anchor=west,left,xshift=-1.8ex,text=olive]{#3} (#1.north west-|T1.south west);
    \end{tikzpicture}
}

\title{Python Basics}
\author{Sumner Evans}
\institute{Mines ACM}

\begin{document}

\begin{frame}{A Small Survey}
    Welcome everyone! I'd like to get to know everyone a bit more and get a feel
    for everyone's prior experience with programming and Python.

    \begin{itemize}[<+->]
        \item What year are you in?
        \item How many of you have programmed in any language before?
        \item How many of you have programmed in \textbf{Python} before?
    \end{itemize}
\end{frame}

\begin{frame}{Overview}
    \setbeamertemplate{section in toc}[sections numbered]
    \tableofcontents[hideallsubsections]
\end{frame}

\section{What is Python?}

\begin{frame}{A Bit of History}
    \begin{itemize}[<+->]
        \item Python first appeared in early 1991. \emph{This means that Python
            is older than Java and Ruby.}
        \item Guido van Rossum (GvR, the creator of Python) designed his
            language with \textbf{emphasis on readability}.
        \item Python was named after \emph{Monty Python's Flying Circus}.
        \item The language quickly gained popularity because of its appeal to
            long-time UNIX/C hackers\footnote{See the Jargon File on hackers
            vs.  crackers}.
    \end{itemize}
\end{frame}

\begin{frame}{Why Learn Python?}
    \begin{itemize}[<+->]
        \item Python is a \textit{general purpose, multi-paradigm} language
            meaning that it is very flexible and can be used in many different
            scenarios.
        \item Some of the main applications of Python in industry are web
            programming, data science, machine learning, automation scripting.
        \item Python is an easy language to learn.
        \item Python runs anywhere, and generally requires little setup compared
            to other languages.
    \end{itemize}
\end{frame}

\begin{frame}{A Note on Python 2 and Python 3}
    There are two main versions of Python: Python 2 and Python 3. As of earlier
    this year, Python 2 is no longer supported, so nobody should use it.
    Unfortunately, many projects and operating systems have not gotten with the
    times and are still reliant on Python 2.

    \pause
    Python 3 has many major advantages over Python 2 as it fixes many annoying
    inconsistencies with the older version.

    \pause
    For the purposes of this presentation, we will be talking \emph{strictly of
    Python 3}.
\end{frame}

\section{Programming Basics in Python}

\begin{frame}{Follow Along}
    You can either install Python on your machine or use an online Python
    environment such as \url{https://repl.it/languages/Python3}.

    Most of the things we will cover today can be done directly in the REPL
    (read-evaluate-print-loop) on the right, however you may want to write code
    in the file on the left and run it.
\end{frame}

\begin{frame}[fragile]{Storing Data}
    At its core, programming is about storing and manipulating \textit{data}.

    \pause
    In almost every programming language, there is a concept of a
    \textbf{variable} which \textit{stores} data.

    \pause
    In Python, you can create a variable using the following syntax:
    \begin{minted}{python}
        name = "Sumner"
        age = 22
        likes_acm = True
    \end{minted}

    \textit{If you have ever programmed in a language such as Java, you may
    notice that there is no special keyword for declaring a variable.}
\end{frame}

\begin{frame}[fragile]{Showing the Data}
    Storing data isn't any good unless you can actually use it for something
    useful. \pause One of the most basic things we can do with the data stored
    is print it out to the console.

    To print anything in Python, use the \texttt{print} function:
    \begin{minted}{python}
        name = "Sumner"
        age = 22
        print(name)
        print(age)
    \end{minted}

    \pause
    If you want to print multiple things at once, you can separate them with a
    comma:
    \begin{minted}{python}
        print(name, age)
    \end{minted}
\end{frame}

\begin{frame}[fragile]{Getting Data From the User}
    Often, we want to get some input from the user and store it in a variable.
    \pause To do this in Python, we use the \texttt{input} function.

    \begin{minted}{python}
        name = input()
        print("Hello", name)
    \end{minted}

    \pause
    We can also optionally include prompt text:
    \begin{minted}{python}
        age = int(input("How old are you? "))
        print("In one year, you will be", age + 1, "years old")
    \end{minted}

    \pause
    What do you think the difference between \texttt{input()} and
    \texttt{int(input())} is?
\end{frame}

\begin{frame}{What Sorts of Data Can We Store?}
    There are many different \textit{types} of data that we can store in Python.
    Here are the most basic data types (primitives):

    \begin{itemize}
        \item \texttt{bool} --- either \texttt{True} or \texttt{False}
        \item \texttt{int} --- an integer
        \item \texttt{float} --- a real number\footnote[frame]{Not all real
            numbers can be represented as a floating point number, but that's
            not normally important.}
        \item \texttt{string} --- a sequence of characters\footnote[frame]{Note
            that unlike other languages, there is no \texttt{char} datatype.
            Chars are just one-character strings.}
    \end{itemize}
\end{frame}

\begin{frame}[fragile]{Manipulating Data: Assigning a New Value to a Variable}
    Having variables to store is nice, but a lot of times we want to modify the
    value stored in the variable!

    To do that, we use a very similar syntax to defining variables:
    \begin{minted}{python}
        name = "Sumner"  # declares a variable and gives it an
                         # initial value
        print(name)

        name = "Jack"    # assigns the value "Jack" to the "name"
                         # variable
        print(name)
    \end{minted}
\end{frame}

\begin{frame}{Manipulating Data: Basic Operations}
    Similar to how you can perform operations on variables in algebra, you can
    perform operations on variables. Here are some basic operations on primitive
    data types in Python:

    \begin{itemize}
        \item \texttt{+}, \texttt{-}, \texttt{*}, \texttt{/}, \texttt{//}: add,
            subtract, multiply, divide, integer division.
        \item \texttt{**}: exponentiate ($3^8$ would be written \texttt{3**8}).
        \item \texttt{<}, \texttt{>}, \texttt{==}: tests if two numbers are less
            than, greater than, equal to each other, respectively.
    \end{itemize}
\end{frame}

\begin{frame}[fragile]{Try it Yourself!}
    \textbf{Try creating a few different variables of various types and
    performing operations on them.}

    A few examples to get you started:
    \begin{minted}{python}
        pi = 3.14159265
        r = 10
        print("diameter =", pi * (r ** 2))

        email = input("Enter your email: ")
        email_again = input("Verify your email: ")
        print(email == email_again)
    \end{minted}
\end{frame}

\begin{frame}[fragile]{Making Decisions: Selection Using \texttt{if}}
    It's all well and good that we can compare numbers and get a boolean value,
    but we need to make \textit{decisions} with that information. That's where
    \textbf{\texttt{if} statements} come in.

    \pause
    The syntax for \texttt{if} statements in Python is:
    \begin{minted}{python}
        if condition1:
            # do whatever is in this indented section if condition1
            # is True
        elif condition2:
            # do whatever is in this indented section if condition2
            # is True
        else:
            # do whatever is in this indented section otherwise
    \end{minted}

    You can have arbitrary many \texttt{elif} blocks (including no elif blocks).
    It is also not necessary to have an \texttt{else} block.
\end{frame}

\begin{frame}[fragile]{Making Decisions: Selection Using \texttt{if}: Example}
    Here is a simple example of an \texttt{if} statement at work:
    \begin{minted}{python}
        likes_python = input("Do you like Python? (y/n)")
        if likes_python == "y":
            print("You like Python!")
        elif likes_python == "n":
            print("You don't like Python :(")
        else:
            print("I don't know if you like Python...")
    \end{minted}
\end{frame}

\begin{frame}[fragile]{Making Decisions: Selection Using \texttt{if}: Your Turn!}
    \textbf{Now it's your turn!} Try to write some Python which does the
    following:

    \begin{enumerate}
        \item Creates a variable with a secret number of your choice.
        \item Asks the user to guess a number.
        \item Tells the user if their guess is above, below, or equal to the
            secret number.
    \end{enumerate}

    \textit{Extra Credit:} Look up the documentation for the
    \texttt{random.randint} function and see if you can make the secret number
    random.
\end{frame}

\begin{frame}[fragile]{Doing Things Many Times: \texttt{for} loops}
    Often, we might want to do the same thing (or a similar thing) multiple
    times. In these situations, we need \textit{loops}.

    \pause
    The most common type of loop in Python is the \texttt{for} loop. Unlike
    other languages, Python's \texttt{for} loop is a \textit{range-based for
    loop}:

    \begin{minted}{python}
        for x in <iterable>:
            # do what is in this indented block
    \end{minted}

    Whatever is in the \textit{indented} section will be run for every value in
    the iterable.
\end{frame}

\begin{frame}[fragile]{Doing Things Many Times: \texttt{for} loops: Example}
    There are a lot of things in Python are iterable. We will see more of them
    soon, but here's an easy example to get you started:

    \begin{minted}{python}
        for x in range(20):
            print(x)
    \end{minted}

    Any ideas what this will print?

    \pause
    The syntax for range is \texttt{range(start, stop, step)}
    \begin{itemize}
        \item \texttt{start} is the number to start on
        \item \texttt{stop} is the number to stop \textit{before}
        \item \texttt{step} is the amount to increment each time
    \end{itemize}

    Both \texttt{start} and \texttt{step} are optional and if omitted, they will
    be assumed to be \texttt{0} and \texttt{1}, respectively.
\end{frame}

\begin{frame}[fragile]{Doing Things Many Times: \texttt{while} loops}
    Sometimes we don't know how many times we need to run a block of code. In
    these cases, we can use a \texttt{while} loop.

    \pause
    The syntax for a \texttt{while} loop is:
    \begin{minted}{python}
        while <condition>:
            # do what is in this indented block
    \end{minted}
\end{frame}

\begin{frame}[fragile]{Doing Things Many Times: \texttt{while} loops: Example}
    What do you think this snippet of code will do?

    \begin{minted}{python}
        i = 0
        while i < 10:
            print(i)
            i = i + 1
    \end{minted}
\end{frame}

\begin{frame}[fragile]{Doing Things Many Times With Loops: Your Turn!}
    \textbf{Now it's your turn!} Try to write some Python which does the
    following (this should feel similar to what you did with the \texttt{if}
    statement):

    \begin{enumerate}
        \item Creates a variable with a secret number of your choice.
        \item Asks the user to guess a number.
        \item Tells the user if their guess is above, below, or equal to the
            secret number.
        \item \textbf{Keeps asking the user to guess until they get the right
            number.}
    \end{enumerate}

    \textit{Extra Credit:} Look up the documentation for the
    \texttt{random.randint} function and see if you can make the secret number
    random.
\end{frame}

\section{Containers for Data: Abstract Data Types}
\begin{frame}{What is an Abstract Data Type?}
    Often, we need to store multiple pieces of data together in a \textit{data
    structure}. Some examples of when this would be useful are:

    \begin{itemize}
        \item Storing a list of all of the students in a class.
        \item Storing the grades associated with each student in a class.
    \end{itemize}

    The reason they are called abstract data types is that you don't need to
    know how they are implemented, you just need to know how they work.
\end{frame}

\begin{frame}{What is an Abstract Data Type?}
    In Python, we have four main data structures:
    \begin{itemize}
        \item \textbf{Lists} --- store an \textit{ordered} set of elements that
            \textit{can} be changed.
        \item \textbf{Tuples} --- store an \textit{ordered} set of elements that
            \textit{cannot} be changed.
        \item \textbf{Sets} --- store an \textit{unordered},
            \textit{deduplicated} set of elements that \textit{can} be changed.
        \item \textbf{Dictionaries} --- store an unordered set of
            mapping/associations of \textit{keys} to \textit{values}.
    \end{itemize}
\end{frame}

\begin{frame}[fragile]{Lists}
    Lists store an ordered set of elements that can be changed. You can declare
    a list \textit{literal} like this:

    \begin{minted}{python}
        [1, 2, 3]
    \end{minted}

    \pause
    If you want to \textit{access} a specific element of the list, you can
    \textit{index} into the list like so:
    \begin{minted}{python}
        my_list = [1, 2, 3]
        print(my_list[0])
        my_list[0] = 4
        print(my_list)
    \end{minted}
    \textbf{Note:} indexes start at \texttt{0} (they are zero-indexed).

    \pause
    If you want to add to a list, you can use the \texttt{append} function:
    \begin{minted}{python}
        my_list = [1, 2, 3]
        my_list.append(4)
        print(my_list)
    \end{minted}
\end{frame}

\begin{frame}[fragile]{Tuples}
    Tuples store an ordered set of elements that cannot be changed. You can
    declare a tuple \textit{literal} like this:

    \begin{minted}{python}
        (1, 2, 3, 4)
    \end{minted}

    \pause
    Like lists, you can access individual elements of the tuple in the same way
    as a list.
    \begin{minted}{python}
        my_tuple = (1, 2, 3)
        print(my_tuple[0])
    \end{minted}

    \pause
    You cannot modify a tuple, but you can add them together (you can actually
    do this with lists, too!):
    \begin{minted}{python}
        my_tuple1 = (1, 2, 3)
        my_tuple2 = (4, 5, 6)
        print(my_tuple1 + my_tuple2)
    \end{minted}
\end{frame}

\begin{frame}[fragile]{Sets}
    Sets store an \textit{unordered, deduplicated} set of elements that can be
    changed. You can declare a set \textit{literal} like this:

    \begin{minted}{python}
        {1, 2, 3, 2}
    \end{minted}

    Note this is equivalent to \texttt{\{1, 2, 3\}} because of the deduplication.

    \pause
    There are two main operations on sets: \texttt{add} and \texttt{remove}:
    \begin{minted}{python}
        my_set = {1, 2, 3}
        my_set.add(4)
        my_set.remove(1)
        print(my_set)
    \end{minted}
\end{frame}

\begin{frame}[fragile]{Lists, Tuples, and Sets + \texttt{for} Loops}
    Lists, Tuples, and Sets are all iterable, meaning they can be used in a
    \texttt{for} loop:
    \begin{minted}{python}
        my_list = [1, 2, 3]
        for x in my_list:
            print(x)
    \end{minted}

    \pause
    \textbf{Your Turn:} Try and create a list of numbers, then for each element
    of the list, print whether that number is less than, equal to, or greater
    than \texttt{42}.

    \pause
    Now try the same thing with a tuple and a set.
\end{frame}

\begin{frame}[fragile]{Dictionaries}
    Dictionaries store an unordered set of mapping/associations of \textit{keys}
    to \textit{values}. You can declare a set \textit{literal} like this:
    \begin{minted}{python}
        {"key": "value", 3: 2, 3: 8}
    \end{minted}

    Note this is equivalent to \texttt{\{"key": "value", 3: 8\}} because, like
    sets, the \textit{keys} are deduplicated.

    \pause
    You can access the value associated with a key of a dictionary similar to
    how you access an element of a list:
    \begin{minted}{python}
        my_dict = {"key": "value", 3: 8}
        print(my_dict["key"])
    \end{minted}

    You can also use similar syntax to edit entries in the dictionary and add
    new entries.
    \begin{minted}{python}
        my_dict = {"key": "value", 3: 8}
        my_dict["new key"] = 42
        my_dict["key"] = 18
        print(my_dict)
    \end{minted}
\end{frame}

\begin{frame}[fragile]{Dictionaries + \texttt{for} Loops}
    Dictionaries are also iterable but most of the time what you care about
    iterating over are either all of the keys, values, or pairs:
    \begin{minted}{python}
        my_dict = {"key": "value", 3: 8}
        print(my_dict.keys())
        print(my_dict.values())
        print(my_dict.items())
    \end{minted}

    \begin{minted}{python}
        for key, value in my_dict.items():
            print(key, value)
    \end{minted}

    \pause
    \textbf{Your Turn:} Create a dictionary containing only numeric keys and
    values and write a \texttt{for} loop which adds the key to the value for
    each of the pairs.
\end{frame}

\section{Functions}

\begin{frame}[fragile]{What is a Function?}
    A lot of times, we have code that we want to run multiple times, but not
    necessarily in right in a row on the same data structure. In these cases, we
    can use \textit{functions} to organize our code.

    \pause
    You've actually already seen functions! \texttt{print}, \texttt{input},
    \texttt{append}, and \texttt{items} are all functions!

    But we can write our own functions as well.
\end{frame}

\begin{frame}[fragile]{Defining our own Functions}
    The syntax for defining a function is:
    \begin{minted}{python}
        def my_function(parameter1, parameter2, ...):
            # do things involving the parameters
            return something
    \end{minted}

    A few notes:
    \begin{itemize}
        \item You can have as many parameters as you want (including none).
        \item You do not have to return anything. (By default, the function will
            return \texttt{None}.)
    \end{itemize}
\end{frame}

\begin{frame}[fragile]{Defining our own Functions: Example}
    Here's a simple function which adds two numbers together:
    \begin{minted}{python}
        def add(a, b):
            return a + b
    \end{minted}
\end{frame}

\begin{frame}{Functions: Your Turn!}
    \begin{itemize}
        \item Create a function that takes in a list as a parameter and returns
            the list, backwards.
        \item Create a function that takes two tuples representing two
            \texttt{(x, y)} coordinates and return the \textit{Cartesian
            distance} between the two points. The formula is:
            $$\sqrt{(x_2 - x_1)^2 + (y_2 - y_1)^2}$$
            (Look up the \texttt{math.sqrt} function.)
    \end{itemize}
\end{frame}

\section{Classes}

\begin{frame}[fragile]{Classes: The Basics}
    There are a lot of things that you can do with classes, but for Carrier
    Python, most of what we use them for is to organize variables and functions
    into logical containers. Here's an example class:

    \begin{minted}{python}
        class MyClass:
            counter = 0

            def add_to_variable(self, amount):
                self.variable = self.variable + amount
    \end{minted}

    You can then instantiate a class like so:
    \begin{minted}{python}
        my_object = MyClass()
        print(my_object.counter)
        my_object.add_to_variable(4)
        print(my_object.counter)
    \end{minted}
\end{frame}

% \begin{frame}{Classes: Your Turn!}
%     \begin{itemize}
%         \item TODO
%     \end{itemize}
% \end{frame}

\end{document}
% Local Variables:
% TeX-command-extra-options: "-shell-escape"
% End:
